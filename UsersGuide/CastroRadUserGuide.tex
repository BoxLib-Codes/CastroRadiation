\documentclass[11pt,letterpaper]{article}

\usepackage{color}
\usepackage{amsmath}
\usepackage{amssymb}
\usepackage{hyperref}
\usepackage{seqsplit}
\usepackage{fullpage}
\usepackage[normalem]{ulem}

\renewcommand{\vec}[1]{\mbox{\boldmath$#1$}}
\newcommand{\tensor}[1]{\mbox{\boldmath{\ensuremath{\mathsf{#1}}}}}
 
% make the MarginPars look pretty
\setlength{\marginparwidth}{0.75in}
\newcommand{\MarginPar}[1]{\marginpar{\vskip-\baselineskip\raggedright\tiny\sffamily
\hrule\smallskip{\color{red}#1}\par\smallskip\hrule}}

\begin{document}

\title{CASTRO Radiation User's Guide}
\maketitle

\section{Introduction}

In this User's Guide we will describe how to use the CASTRO radiation
module.  Several radiation solvers have been developed.  Here we will
describe two solvers: a gray rdiation hydrodynamics solver and a
multigroup (MG) radiation hydrodynamics solver.  Both solvers adopt
the flux-limited diffusion (FLD) approximation.  The gray solver has a
comoving frame mode and a mixed frame mode, whereas the MG solver uses
the comoving frame approach.  The MG solver can also be used to solve
gray problems.  More details about the formulation and
algorithm can be found in the seriers of CASTRO papers and the CASTRO
User's Guide.

\section{Getting Started}

\subsection{Getting the Code}

\paragraph{BoxLib} BoxLib is distributed via git.  You can download it
by typing
\vspace{5pt}

\verb|git clone https://ccse.lbl.gov/pub/Downloads/BoxLib.git|\\[5pt]
and update it regularly by typing \verb|git pull| in the BoxLib
directory. 

\paragraph{CASTRO} You can download CASTRO via git by typing
\vspace{5pt}

\verb|git clone XXX@battra.lbl.gov:/home/ccse_src/gitroot/Castro.git|\\[5pt]
where {\tt XXX} should be replaced with your user name on battra.  You
can update it regularly by typing \verb|git pull| in the Castro
directory.

\paragraph{CASTRO\_RAD} You need to be in the group {\tt Castro\_radiation}
on {\tt battra.lbl.gov} to have access to CASTRO\_RAD.  If not, contact Ann
Almgren at {\tt ASAlmgren@lbl.gov}.  The environment variable
\verb|CVS_RSH| should be set to \verb|ssh|.  You can set
\verb|CVS_RSH| by typing \verb|export CVS_RSH=ssh| if you use bash. To
download CASTRO\_RAD, you type \vspace{5pt}

\verb|cvs -d :ext:XXX@battra.lbl.gov:/home/ccse_src/cvsroot co CASTRO_RAD|\\[5pt]
where {\tt XXX} should be replaced with your user name on battra.

\paragraph{Hypre} CASTRO radiation solvers use Hypre
(\url{https://computation.llnl.gov/casc/linear_solvers/sls_hypre.html})
for solving linear systems.  So you need to install Hypre on your
machine.

\subsection{Building the Code}

You need to modify:
\begin{itemize}
\item the variable {\tt BOXLIB\_HOME} in the file
  {\tt\seqsplit{CASTRO/Parallel/Castro/Exec/Make.Castro}} to the
  directory for BoxLib (e.g., {\tt\seqsplit{/home/J.Doe/BoxLib}})
\item the variable {\tt CASTRO\_DIR} in the file
  {\tt\seqsplit{CASTRO\_RAD/Castro\_radiation/Exec/Make.Castro}} to
  the Castro directory
  (e.g., {\tt\seqsplit{/home/J.Doe/CASTRO/Parallel/Castro}})
\item the variable {\tt HYPRE\_DIR} in the file
  {\tt\seqsplit{CASTRO\_RAD/Castro\_radiation/Exec/Make.Castro}} to
  the directory for Hypre (e.g., {\tt\seqsplit{/home/J.Doe/hypre}})
\end{itemize}
or you can set them as environment variables of your shell.  \\


Now go to a ``run'' directory, say
{\tt\seqsplit{CASTRO\_RAD/Castro\_radiation/Exec/RadThermalWave}},
edit the file {\tt GNUmakefile}, and set
\begin{itemize}
\item {\tt COMP} = your favorite C++ compiler (e.g., g++)
\item {\tt FCOMP} = your favorite Fortran compiler (e.g., gfortran)
\item {\tt USE\_HYPRE} = {\tt TRUE}
\item {\tt USE\_RAD}   = {\tt TRUE}
\item {\tt RAD\_INTERP}   = {\tt TRUE}
\item {\tt DIM}   = 1 or 2 or 3
\end{itemize}
Then type {\tt make} to generate an executable file.  

\section{EOS, Network, and Opacity}

\paragraph{EOS} CASTRO provides two types of equation of state (EOS):
gamma-law and Helmholtz.  To use the gamma-law EOS, set 
\vspace{5pt}

\verb|EOS_dir := gamma_law_general|\\[5pt]
in the {\tt GNUmakefile}.  The original Helmholtz EOS includes
radiation contribution.  However, for radiation hydrodynamics
calculations, the radition contribution should be taken out of EOS
because radiation has been treated in other places.  To use Helmholtz
EOS, set \vspace{5pt}

\verb|EOS_dir := helmeos_without_rad|\\[5pt]
and do NOT set \verb|EOS_dir| to {\tt helmeos}.  If you have your own
EOS, you can put it in either {\tt\seqsplit{CASTRO/Parallel/Castro/EOS}} or
{\tt\seqsplit{CASTRO\_RAD/Castro\_radiation/EOS}}

\paragraph{Network} Some netwroks are included in the source code.  If
you have your own network, you can put it in either
{\tt\seqsplit{Castro/Networks}} or
{\tt\seqsplit{CASTRO\_RAD/Castro\_radiation/Networks}}.

\paragraph{Opacity} By default, we assume that
\begin{equation}
  \kappa = \mathrm{const}\ \rho^{m} T^{-n} \nu^{p} , \label{eq:kappa}
\end{equation}
where $\kappa$ is either Planck or Rosseland mean absorption
coefficients, $\rho$ is density, $T$ is temperature, $\nu$ is
frequency, and $m$, $n$ and $p$ are constants.  For the gray solver,
$p = 0$.  If Equation~(\ref{eq:kappa}) is sufficient, set \vspace{5pt}

\verb|Opacity_dir := null|\\[5pt]
in {\tt GNUmakefile}.  Otherwise, put your own opacity in
{\tt\seqsplit{CASTRO\_RAD/Castro\_radiation/Opacity}} and set
the input parameter, {\tt radiation.use\_opacity\_table\_module} (see
\S~\ref{sec:opacpars}). 

\section{Input Parameters}

In this section, we list some radiation related parameters that you
can set in an ``inputs'' file.  Here are some important parameters:
\begin{description}
\item[\ \ \ radiation.SolverType] \hfill \\
  Set it to 5 for the gray solver, and 6 for the MG solver.
\item[\ \ \ castro.do\_hydro] \hfill \\
  Usually you want to set it to 1.  If it is set to 0, hydro will be
  turned off, and the calculation will only solve radiation diffusion
  equation.
\item[\ \ \ castro.do\_radiation] \hfill \\
  If it is 0, the calculation will be pure hydro.
\end{description}

Below are more parameters.  For each parameter, the default value is
on the right-hand side of the equal sign. 

\subsection{Flux Limiter and Closure}

\begin{description}
\item[\ \ \ radiation.limiter = 2] \hfill \\
  Possible values are:
  \begin{itemize}
    \item 0\ : No flux limiter
    \item 2\ : Approximate limiter of Levermore \& Pomraning
    \item 12: Bruenn's limiter
    \item 22: Larsen's square root limiter
    \item 32: Minerbo's limiter
  \end{itemize}
\item[\ \ \ radiation.closure = 3] \hfill \\
  Possible values are:
  \begin{itemize}
    \item 0: $f = \lambda$, where $f$ is the scalar Eddington factor
      and $\lambda$ is the flux limiter. 
    \item 1: $f = \frac{1}{3}$ 
    \item 2: $f = 1 - 2 \lambda$ 
    \item 3: $f = \lambda + (\lambda R)^2$, where $R$ is the radiation
      Knudsen number. 
    \item 4: $f = \frac{1}{3} + \frac{2}{3} (\frac{F}{cE})^2$, where
      $F$ is the radiation flux, $E$ is the radiation energy density,
      and $c$ is the speed of light.
  \end{itemize}
% \item[\ \ \ radiation.surface\_average = 2] \hfill \\
%   In the discretization of the diffusion equation, the diffusion
%   coefficients live on cell surfaces.  Thus averaging .....
\end{description}

\subsection{Opacity}
\label{sec:opacpars}

\begin{description}
\item[radiation.use\_opacity\_table\_module = 0] \hfill \\
  For neutrino provides, this parameter is not used.  For photon
  problems, this determines whether the opacity module at {\tt
    Opacity\_dir} (which is set in {\tt GNUmakefile}) will be used to
  compute opacities.  If this is set to 1, the following parameters
  for opacities will be ignored.  
\item[radiation.const\_kappa\_p = -1.0] \hfill \\
  The constant in Equation~(\ref{eq:kappa}) for Planck mean
  opacity. 
\item[radiation.kappa\_p\_exp\_m = 0.0] \hfill \\
  The exponent $m$ in Equation~(\ref{eq:kappa}) for Planck mean
  opacity. 
\item[radiation.kappa\_p\_exp\_n = 0.0] \hfill \\
  The exponent $n$ in Equation~(\ref{eq:kappa}) for Planck mean
  opacity. 
\item[radiation.kappa\_p\_exp\_p = 0.0] \hfill \\
  The exponent $p$ in Equation~(\ref{eq:kappa}) for Planck mean
  opacity. 
\item[radiation.const\_kappa\_r = -1.0] \hfill \\
  The constant in Equation~(\ref{eq:kappa}) for Rosseland mean
  opacity. 
\item[radiation.kappa\_r\_exp\_m = 0.0] \hfill \\
  The exponent $m$ in Equation~(\ref{eq:kappa}) for Rosseland mean
  opacity. 
\item[radiation.kappa\_r\_exp\_n = 0.0] \hfill \\
  The exponent $n$ in Equation~(\ref{eq:kappa}) for Rosseland mean
  opacity. 
\item[radiation.kappa\_r\_exp\_p = 0.0] \hfill \\
  The exponent $p$ in Equation~(\ref{eq:kappa}) for Rosseland mean
  opacity. 
\item[radiation.const\_scattering = 0.0] \hfill \\
  The constant in Equation~(\ref{eq:kappa}) for scattering coefficient.
\item[radiation.scattering\_exp\_m = 0.0] \hfill \\
  The exponent $m$ in Equation~(\ref{eq:kappa}) for scattering coefficient.
\item[radiation.scattering\_exp\_n = 0.0] \hfill \\
  The exponent $n$ in Equation~(\ref{eq:kappa}) for scattering coefficient.
\item[radiation.scattering\_exp\_p = 0.0] \hfill \\
  The exponent $p$ in Equation~(\ref{eq:kappa}) for scattering coefficient.
\item[radiation.kappa\_r\_floor = 0.0] \hfill \\
  Floor for Rosseland mean.
\item[radiation.do\_kappa\_stm\_emission = 0] \hfill \\
  If it is 1, correction for stimulated emission is applied to Planck mean as
  follows
  \begin{equation}
    \kappa = \mathrm{const}\ \rho^{m} T^{-n} \nu^{p}
    [1-\exp{(-\frac{h\nu}{k T})}].
  \end{equation}
\end{description}

\noindent Note that the unit for opacities is $\mathrm{cm}^{-1}$.  For
the gray solver, the total opacity in the diffusion coefficient is the sum
of {\tt kappa\_r} and {\tt scattering}, whereas for the MG solver,
there are two possibilities.  If {\tt const\_kappa\_r} is greater than
0, then the total opacity is set by {\tt kappa\_r} alone, otherwise
the total opacity is the sum of {\tt kappa\_p} and {\tt scattering}. 

\subsection{Boundary}

The following parameters are for radiation boundary in the diffusion
equation. They do not affect hydro boundaries. 
\begin{description}
\item[radiation.lo\_bc] \hfill \\
  Possible values are:
  \begin{itemize}
    \item 101: Dirchlet
    \item 102: Reflect
    \item 103: Neumann
    \item 104: Marshak (vacuum) 
    \item 105: Sanchez-Pomraning (modified Marshak that works with FLD)
  \end{itemize}
\item[radiation.hi\_bc] \hfill \\
  See {\tt radiation.lo\_bc}.
\end{description}

\subsection{Convergence}

For the gray solver, there is only one iteration in the scheme,
whereas for the MG solver, there are two iterations with an inner
iteration embedded inside an outer iteration.  In the following, the
iteration in the gray solver will also be referred as the outer
iteration for convenience.  The parameters for the inner iteration are
irrelevant to the gray solver.  However, the inner iteration is used
for a gray problem solved by the MG solver.  

\begin{description}
\item[radiation.maxiter = 50] \hfill \\
  Maximal number of outer iteration steps. 
\item[radiation.miniter = 1] \hfill \\
  Minimal number of outer iteration steps. 
\item[radiation.reltol = 1.e-6] \hfill \\
  Relative tolerance for the outer iteration.
\item[radiation.abstol = 0.0] \hfill \\
  Absolute tolerance for the outer iteration.
\item[radiation.maxInIter = 30] \hfill \\
  Maximal number of inner iteration steps. 
\item[radiation.minInIter = 1] \hfill \\
  Minimal number of inner iteration steps. 
\item[radiation.relInTol = 1.e-4] \hfill \\
  Relative tolerance for the inner iteration.
\item[radiation.absInTol = 0.0] \hfill \\
  Absolute tolerance for the inner iteration.  
\item[radiation.convergence\_check\_type = 0] \hfill \\
  For the MG solver only.  This specifiy the way of checking the
  convergence of an outer iteration.  Possible values are
  \begin{itemize}
    \item 0: Check $T$, $Y_e$, and the residues of the equations for
      $\rho e$ and $\rho Y_e$
    \item 1: Check $\rho e$
    \item 2: Check the residues of the equations for $\rho e$ and $\rho Y_e$
    \item 3: Check $T$ and $Y_e$
  \end{itemize}
\end{description}


\subsection{Parameters for Gray Solver}
\label{sec:graypar}

\begin{description}
\item[radiation.comoving = 1] \hfill \\
  Do we use the comoving frame approach?
\item[radiation.Er\_Lorentz\_term = 1] \hfill \\
  If the mixed-frame approach is taken, this parameter decides whether
  Lorentz transformation terms are retained. 
\item[radiation.delta\_temp = 1.0] \hfill \\
  This is used in computing numerical derivativas with respect to $T$.
  So it should be a small number compared with $T$, but not too small.  
\item[radiation.update\_limiter = 1000] \hfill \\
  Stop updating flux limiter after {\tt update\_limiter} iteration steps. 
\item[radiation.update\_planck = 1000] \hfill \\
  Stop updating Planck mean opacity after {\tt update\_planck} iteration steps. 
\item[radiation.update\_rosseland = 1000] \hfill \\
  Stop updating Rosseland mean opacity after {\tt update\_rosseland} iteration steps. 
\end{description}

\subsection{Grouping in the MG Solver}

We provide two methods of setting up groups based upon logarithmic
spacing.  In both methods, you must provide: 
\begin{description}
\item[radiation.nGroups] \hfill \\
  Number of groups. 
\item[radiation.lowestGroupHz] \hfill \\
  Frequency of the lower bound for the first group.
\end{description}

In addition, if the parameter {\tt groupGrowFactor} is provided, then
the first method will be used, otherwise the second method will be
used.  In the first way, you must also provide {\tt firstGroupWidthHz}
(the width of the first group).  The width of other groups is set to
be {\tt groupGrowFactor} times the width of its immediately preceding
group.  In the second way, you must provide {\tt highestGroupHz} as
the upper bound of the last group.  It should be noted that {\tt
  lowestGroupHz} can be 0 in the first method, but not the second
method.  However, when we compute the group-integrated Planck
function, the lower bound for the first group and the upper bound for
the last group are assumed to be 0 and $\infty$, respectively.

The MG solver can also be used to solve gray problems, if you set {\tt
  radiation.nGroups = 1}.  In that case, you do not need to set other
grouping parameters and the frequency range will be assumed to be $[0,
\infty]$. It should be noted that the MG solver is not as efficient as
the real gray solver for most gray problems, because it uses more
iterations.

\subsection{Parameters for MG Solver}
\label{sec:mgpar}

\begin{description}
\item[radiation.delta\_e\_rat\_dt\_tol = 100.0] \hfill \\
  Maximally allowed relative change in $e$ during one time step.
\item[radiation.delta\_T\_rat\_dt\_tol = 100.0] \hfill \\
  Maximally allowed relative change in $T$ during one time step.
\item[radiation.delta\_Ye\_dt\_tol = 100.0] \hfill \\
  Maximally allowed absolute change in $Y_e$ during one tim estep.
\item[radiation.fspace\_advection\_type = 2] \hfill \\
  Possible value is 1 or 2.  The latter is better.
\item[radiation.integrate\_Planck = 1] \hfill \\
  If 1, integrate Planck function for each group.  For the first
  group, the lower bound in the integration is assumed to be 0 no
  matter what the grouping is.  For the last group, the upper bound in
  the integration is assumed to be $\infty$.
\item[radiation.matter\_update\_type = 0] \hfill \\
  How to update matter.  0 is proabaly the best.
\item[radiation.accelerate = 2] \hfill \\
  The inner iteration of the MG solver usually requires an
  acceleration scheme.  Choices are
  \begin{itemize}
    \item 0: No acceleration
    \item 1: Local acceleration
    \item 2: Gray acceleration
  \end{itemize}
\item[radiation.skipAccelAllowed = 0] \hfill \\
  If it is set to 1, skip acceleration if it does not help. 
\item[radiation.n\_bisect = 1000] \hfill \\
  Do bisection for the outer iteration after {\tt n\_bisec} iteration steps.
\item[radiation.use\_dkdT = 1] \hfill \\
  If it is 1, $\frac{\partial \kappa}{\partial T}$ is retained in the
  Jacobi matrix for the outer (Newton) iteration.  
\item[radiation.update\_opacity = 1000] \hfill \\
  Stop updating opacities after {\tt update\_opacity} outer iteration steps.
\item[radiation.inner\_update\_limiter = 0] \hfill \\
  Stop updating flux limiter after {\tt inner\_update\_limiter} inner
  iteration steps.  If it is 0, the limiter is lagged by one outer
  iteration.  If it is -1, the limiter is lagged by one time step.  If
  the inner iteration has difficulty in converging, setting this
  parameter it to -1 can help.  Since the flux limiter is only a
  kludge, it is justified to lag it. 
\end{description}

\subsection{Verbosity and I/O}
\label{sec:bothpar}

\begin{description}
\item[radiation.v = 0] \hfill \\
  Verbosity
\item[radiation.verbose = 0] \hfill \\
  Verbosity
\item[radiation.Test\_Type\_lambda = 0] \hfill \\
  The gray solver only.  If 1, save flux limiter.
\item[radiation.Test\_Type\_Flux = 0] \hfill \\
  If 1, save the total radiation flux. 
\end{description}

\subsection{Linear System Solver}
\label{sec:hypre}

There are a number of choices for the linear system solver.  The
performance of the solvers usually depends on problems and the
computer.  So it is worth trying a few solvers to find out which one
is best for your problem and computer.

\begin{description}
\item[radsolve.level\_solver\_flag] \hfill \\
  Setting this to 109 (GMRES using Struct SMG/PFMG as preconditioner)
  should work reasonably well for most problems. 
\item[hmabec.struct\_flag] \hfill \\
  This parameter is for {\tt level\_solver\_flag} = 109. Set this to 0
  (SMG) for 1D, 1 (PFMG) for multi-D.
\item[radsolve.maxiter = 40] \hfill \\
  Maximal number of iteration in Hypre.
\item[radsolve.reltol = 1.e-10] \hfill \\
  Relative tolerance in Hypre
\item[radsolve.abstol = 0] \hfill \\
  Absolute tolerance in Hypre
\item[radsolve.v = 0] \hfill \\
  Verbosity
\item[radsolve.verbose = 0] \hfill \\
  Verbosity
\item[habec.verbose = 0] \hfill \\
  Verbosity for {\tt level\_solver\_flag} $<$ 100
\item[hmabec.verbose = 0] \hfill \\
  Verbosity for {\tt level\_solver\_flag} $>=$ 100
\end{description}

\end{document}
